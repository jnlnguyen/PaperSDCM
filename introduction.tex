\section*{Introduction}
\textbf{flow not good, to be revised, add lit}

Plastic materials are an integral part of modern society due the sheer endless application possibilities.
Yet, due to the lack of recycling measurements vast amounts of plastic waste end up uncontrollably in our environment.
Numerous studies revealed to us the impact of this unstoppable plastic pollution: microplastics derived from plastic waste can be found marine lifeforms.
Only recently, further studies found microplastics in human stool and the human placenta with unknown long-term effects. 

To trace back the different pathways of microplastic into the human body, we need identification techniques that allow us to identify and classify plastic waste at different stages.
Here, Raman and FTIR spectroscopy are commonly used in plastic pollution studies due to the availablity of commercial systems.
At the same time, plastic pollution samples are highly diverse due to environmental influences and the sheer endless combinations to produce plastic.
Due to the physical limitations of each aformentioned technqiue, it is forseeable that each technique only work for a subset of plastic waste.
Consquently, additional techniques are required to cover in particular those plastic waste types are currently not covered.
A recent study by Ornik et al. [XXXREF] demonstrated that photoluminescence is a suitable technique to identify plastics from other materials that occur in the riverine environment.

Compared to the previous techniques, photoluminescence (PL) stands out for its simplicity.
A basic setup consists of only two components, namely a monochromatic laser and spectrometer, which makes it a globally accessible technique for microplastic identification.
However, the simplicity of the basic setup also raises questions about the comparability of different spectral data.
Even if a sample is measured by two setups with identical hardware, the acquired spectra can look different because of different alignments, sample sites or even scientific experiences.
Thus, while measurements should always be taken at laboratory conditions, this cannot be always fulfilled and raises the question for other identification methods that can be used instead.
One possible solution is to take advantage of the fact that large sets of data can be generated due to the simplicity of the setup.
Once integrated in a library, computer algorithms and models can be developed to help unraveling the origins of the plastic sample.

Today, we have a choice over a vast amount of computational methods that could be used for plastic classification.
Amongst them, machine learning models are the most popular ones with many established methods available to the public.
Generating a machine learning model consists of two steps: first, we identify patterns on a selected dataset and second, we select a learning model to use these patterns and identify the plastic samples.
Since different combinations of methods can be used to do both steps, it is not clear if the chosen method can work with high accuracies, i.e. the probability that a model based prediction is correct, in the presence of data heterogeneities due to experimental variations.

Here, we look at two different methods to identify patterns.
The first one uses all information from a single spectrum while the second one uses a method known as signal dissection by correlation maximization (SDCM).
The latter has the advantage that physically meaningful patterns can be extracted.
Our study shows that, machine learning models based on SDCM are more robust towards experimental heterogoneities.