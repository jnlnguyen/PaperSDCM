\section*{Introduction}
\textbf{Für mich JN:flow not good, to be revised, add lit}

The increasing plastic pollution is a potential health hazard of our modern society.
Once plastic waste enters our environment, they can persist for centuries as most plastic types are resistant to natural degradation processes.
Environmental influences, however, can cause plastic waste deintegrate into micron-sized particles commonly refered to as microplastics.
Today, microplastics have already been detected on numerous sites which suggests that most species are exposed to microplastics on a daily basis that can then be absorbed or ingested.
Since 2020, we also know that microplastics can be found in humans when a study by Ragusa et al. detected microplastics in the human placenta.
To evaluate the risk of microplastics in life forms, the vast amount of sources for microplastics must be accounted for.
Here, detection techniques to identify and classify plastic waste at different stages are indispensable to trace a particle back to its source.

The common tools to analyse extracted plastic samples from the environment are Raman and FTIR spectroscopy.
Due to the physical limitations of each aformentioned technqiue, it is forseeable that each technique only work for a subset of plastic waste.
However, since plastic pollution samples are highly diverse due to environmental influences and the sheer endless combinations to produce plastic.

Consquently, additional techniques are required to cover in particular those plastic waste types are currently not covered.
A recent study by Ornik et al. \cite{Ornik2020} demonstrated that photoluminescence is a suitable technique to identify plastics from other materials that occur in the riverine environment.

Compared to the previous techniques, photoluminescence (PL) stands out for its simplicity.
A basic setup consists of only two components, namely a monochromatic laser and spectrometer, which makes it a globally accessible technique for microplastic identification.
However, the simplicity of the basic setup also raises questions about the comparability of different spectral data.
Even if a sample is measured by two setups with identical hardware, the acquired spectra can look different because of different alignments, sample sites or even scientific experiences.
Thus, while measurements should always be taken at laboratory conditions, this cannot be always fulfilled and raises the question for other identification methods that can be used instead.
One possible solution is to take advantage of the fact that large sets of data can be generated due to the simplicity of the setup.
Once integrated in a library, computer algorithms and models can be developed to help unraveling the origins of the plastic sample.

Today, we have a choice over a vast amount of computational methods that could be used for plastic classification.
Amongst them, machine learning models are the most popular ones with many established methods available to the public.
Generating a machine learning model consists of two steps: first, we identify patterns on a selected dataset and second, we select a learning model to use these patterns and identify the plastic samples.
Since different combinations of methods can be used to do both steps, it is not clear if the chosen method can work with high accuracies, i.e. the probability that a model based prediction is correct, in the presence of data heterogeneities due to experimental variations.

Here, we look at two different methods to identify patterns.
The first one uses all information from a single spectrum while the second one uses a method known as signal dissection by correlation maximization (SDCM).
The latter has the advantage that physically meaningful patterns can be extracted.
Our study shows that, machine learning models based on SDCM are more robust towards experimental heterogoneities.

Discovery of micrplastics in human body
Nobody likes plastic in our body.
What kind of plastics are they
Where do they come from?

plastics in human
need for detection of plastics
techniques
PL
Simple but can vary
Robust method which works with signatures
use SDCM