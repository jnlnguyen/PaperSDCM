\section*{Introduction}
Our planet is drowning in plastic litter that finds its way into our food and our body over time.
An estimation in 2010 found that at least 4.8 million tons of plastic litter has already entered our ocean\cite{Jambeck2015a}.
Without an effective waste management plan this amount will continue to grow every year\cite{Jambeck2015a, Geyer2017}. 
Once in our environment, plastic litter can persist for decades as most types are resistant to natural degradation processes\cite{Chamas2020}.
Environmental influences, however, can cause it to disintegrate into micron-sized particles commonly known as microplastics\cite{Thompson2004, Julienne2019, Zhang2021, Song2017, Duis2016}.
Almost invisible to the eye, they are impossible to avoid: we find them in nearly every corner of our planet\cite{Duis2016,Chiba2018,Napper2020,Eerkes-Medrano2015,Andrady2011,Allen2019a}, in animals \cite{Barboza2020,Haave2021, Jamieson2019} and even in our food\cite{Koelmans2019,Zhang2020a}.
Since 2021, we also have the first evidence of microplastics in our organs, when Ragusa et al. detected them in the placenta\cite{Ragusa2021}.

Plastic litter is highly diverse due to environmental influences and the sheer endless possibilities to produce plastic with desired material properties.
Therefore, to evaluate their detrimental effects on animals and humans, we need a deeper insight on the sample's origins\cite{Prata2020, Campanale2020, Lim2021}.
Here, tools to detect and classify plastic litter at different stages play an indispensable role.
Studies on plastic pollution commonly use Raman and Fourier transform infrared (FTIR) spectroscopy solutions to analyse plastic samples\cite{Prata2019,Loder2015,Sun2019,Zhang2020b}.
Both techniques, however, come with physical limitations\cite{Araujo2018a, Loder2015,Xu2019} and hence, some plastic litter types may remain undetected to date.

Most recently, Ornik et al. \cite{Ornik2020} used photoluminescence (PL) spectroscopy for plastic identification.
By comparing spectral intensity ratios between different samples, they sucessfully distinguished plastics from non-plastic samples from the riverine and marine environment.
Such an identification method, however, may not be reproducible since measurements depend on different experimental factors such as hardware and sample alignments or even scientific experiences.
To capture and quantify all these influences is impractical as they may only apply to the examined setup.
%Nevertheless, it raises the question if there are subsets in the spectra that can be used for the identification while being robust against such experimental variations.
One possible alternative is to measure a part of these variations and integrate them in a spectral library.
Once implemented, algorithms and mathematical models can help unraveling the plastic sample's origin.

For predictions that account for data variations, machine learning (ML) models are prominent candidates.
We can generate a plastic waste prediction model by applying a classification method on the labeled spectral dataset.
The model's performance partially depends on the number of input variables in a dataset.
In our case, this is the measured intensity for each channel of our spectrometer.
To improve the performance, it is common practice to reduce the number of variables while retaining the essential information in the data\cite{Aggarwal2001,Fodor2002}.
An analysis of the stripped-down dataset can reveal signatures for groups of samples with common properties, such as, plastic type or colour. 
The efficiency to identify and capture these signatures depends on the selected data transformation algorithm.
Here, a recently published technique termed as signal dissection by correlation maximization (SDCM) stands out which successfully discovered new hidden signatures in a gene expression dataset \cite{Grau2019}.
This is particularly attractive for PL-based plastic litter detection as it could identify unique subspectra that only occur in specific samples or sample types.

In the present report, we make two proposals for the PL-based sample identifications:
first, ML is the method of choice for identification; and second, SDCM has a superior ability to identify unique sample type signatures.
The latter is advanteagous as it improves our ability to track down the sample origin to a single source.
For this purpose, we generate ML models with different classification methods and data transformation algorithms.
By evaluating model's performance metrics, e.g. the probability that a model-based prediction is correct, and the extracted signatures we find evidence that SDCM-based ML models outperform their counterparts.