\section*{Introduction}
Our planet is drowning in plastic litter that can sneakily enter our body over time.
An estimation showed that in 2010 at least 4.8 million tons of plastic litter has already entered our ocean\cite{Jambeck2015a}; a volume that will increase without an effective waste management plan\cite{Jambeck2015a, Geyer2017}. 
Once in the wild, plastic can persist for decades as most types are resistant to natural degradation processes\cite{Chamas2020}.
Environmental influences, however, can cause it to disintegrate into micron-sized particles commonly known as microplastics\cite{Thompson2004, Julienne2019, Zhang2021, Song2017, Duis2016}.
Almost invisible to the eye, they have now been detected on nearly every corner of our planet\cite{Duis2016,Chiba2018,Napper2020,Eerkes-Medrano2015,Andrady2011,Allen2019a}, in animals \cite{Barboza2020,Haave2021, Jamieson2019} and even in our food\cite{Koelmans2019,Zhang2020a}.
Since 2021, we also have the first evidence that microplastics are present in humans, when Ragusa et al. detected microplastics in the human placenta\cite{Ragusa2021}.

Plastic litter is highly diverse due to environmental influences and the sheer endless possibilities to produce plastic with desired material properties.
Therefore, to evaluate their detrimental effects on animals and humans, we need a deeper insight on the samples origins\cite{Prata2020, Campanale2020, Lim2021}.
Here, tools to detect and classify plastic litter at different stages play an indispensable role.
Studies on plastic pollution commonly use Raman and Fourier transform infrared (FTIR) spectroscopy solutions to analyse plastic samples\cite{Prata2019,Loder2015,Sun2019,Zhang2020b}.
Both techniques, however, come with physical limitations\cite{Araujo2018a, Loder2015,Xu2019} and hence, we only cover a subset of plastic waste types out there.

Most recently, Ornik et al. \cite{Ornik2020} used photoluminescence (PL) spectroscopy for plastic identification.
By comparing the intensity ratios in the PL spectrum of different samples, they sucessfully distinguished plastics from non-plastic samples in the riverine and marine environment.
Such an identification method, however, may not be reproducible since a measurement can change with different experimental factors such as hardware alignments, sample sites or even scientific experiences.
On the other hand, it is impractical to capture and quantify these influences since it is impossible to account for all possible factors.
Nevertheless, it raises the question if there are subsets in the spectra that can be used for the identification while being robust against experimental variations.
One possible solution is to capture a part of these variations and integrate them in a spectral library.
Once implemented, algorithms and mathematical models help unraveling the origins of the plastic sample.

For predictions that account for data variations, machine learning (ML) models are suitable candidates.
To generate a plastic waste prediction model, we apply a selected learning method on the labeled spectral dataset.
The model's performance partially depends on the number of input variables in a dataset, which, in our case, is the intensity for each channel of our spectrometer.
To improve the performance, it is common practice to reduce the number of variables while retaining the essential data information \cite{Aggarwal2001,Fodor2002}.
A careful choice of the transformation technique allows a physical interpretation of the results and thus, a better understanding of the dataset.
For such an application, a recently published technique termed as signal dissection by correlation maximization (SDCM) stands out which successfully discovered new signatures in a gene expression dataset \cite{Grau2019}.
This is particularly attractive for PL-based plastic litter detection as it can identify type specific subspectra that not only allow a physical interpretation but are also robust against experimental variations.

In this report, we demonstrate that SDCM is suitable to generate robust PL-based plastic litter identification models.
To demonstrate this, we look at two sets of ML models that are either based on a SDCM-transformed PL dataset or an untransformed dataset.
By comparing model's accuracies, i.e. the probability that a model based prediction is correct, we find consistently higher values for SDCM-based ML models than their counterparts.
This underlines the robustness of SDCM-transformed PL datasets against experimental variations.