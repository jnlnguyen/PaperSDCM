\section*{Introduction}
The increasing plastic pollution is a potential health hazard of our modern society.
Once plastic waste enters our environment, they can persist for decades as most plastic types are resistant to natural degradation processes.
Environmental influences, however, can cause plastic waste disintegrate into micron-sized particles commonly referred to as microplastics.
To date, microplastics have already been detected on numerous sites which suggests that most species are exposed them on a daily basis that can then be absorbed or ingested.
Since 2020, we also know that microplastics can be found in humans when a study by Ragusa et al. detected microplastics in the human placenta.
To evaluate the health risk of microplastics, their different sources must be taken into account.
Here, detection techniques to identify and classify plastic waste at different stages are indispensable to trace a particle back to its source.

Plastic waste samples are highly diverse due to environmental influences and the sheer endless possibilities to produce plastic with desired material properties.
Studies on environmental pollution commonly use Raman and Fourier transform infrared (FTIR) spectroscopy solution to analyse plastic samples.
Both techniques, however, come with physical limitations and hence, only a subset of plastic waste types can be covered.
As a result, additional techinques for plastic waste identification are desirable.

Most recently, Ornik et al. \cite{Ornik2020} used photoluminescence (PL) spectroscopy for plastic identification.
They demonstrated that by using intensity ratios different plastic samples can be distinguished from other samples in the riverine environment.
This identification method, however, may not be reproducible since a PL spectrum can change with different experimental factors such as hardware alignments, sample sites or even scientific experiences.
At the same time, it is impractical to capture and quantify these influences since the number of possible factors is likely to be huge.
An alternative solution is to take large sets of spectra that capture only a part of variations due to changes in the measurement and integrate them in a spectral library.
Once implemented, algorithms and mathematical models help unraveling the origins of the plastic sample.
For such applications, machine learning (ML) models are suitable candidates as they account for data variations in their predictions.

To generate a plastic waste prediction model, we apply a selected learning method on the labeled spectral dataset.
The model's performance partially depends on the number of input variables in a dataset, i.e. the dataset's dimensionality.
For example, in our study this would be the intensity for each channel of our spectrometer.
To improve the performance of the model, it is common practice to reduce the number of input variables while retaining the essential data information.
A careful choice of the transformation technique allows a physical interpretation of the results and thus, a better understanding of the dataset.
Here, a recently published technique termed as signal dissection by correlation maximization (SDCM) stands out which has been successfully demonstrated on a medical dataset \cite{Grau2019}.
The application of this method on PL spectra for plastic waste prediction has not been applied yet.

In this study, we demonstrate that SDCM can produce a low-dimensional, physically interpretable dataset that can be used to generate prediction models that are robust against experimental heterogeneities


to do both steps, it is not clear which combination works with high accuracies, i.e. the probability that a model based prediction is correct, for our outlined problem.



The first one uses all information from a single spectrum while the second one uses a method known as signal dissection by correlation maximization (SDCM).
The latter has the advantage that physically meaningful patterns can be extracted.
Our study shows that, machine learning models based on SDCM are more robust towards experimental heterogoneities.

svc
random forest