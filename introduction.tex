\section*{Introduction}

-	Text on techniques for plastic identification (Koch)
-	One technique: Photoluminescence
-	Ornik et al.: Identification of plastics with intensity ratios
-	PL spectra of a single sample can vary
o	Setup related factors: equipment, alignment. Source
o	Identifications based on intensities at specific wavelengths lead to errors
-	Central question: Is there a robust plastic identification method with heterogeneous PL spectral data?
o	Robust = achieve high identification accuracies despite data heterogeneities
o	Look ONLY at heterogeneities due to experimental variations
-	Use machine learning (ML)
o	The ML model generation consists of two steps:
	Provide the data for model training
	Select a learning method to train the model, e.g. neuronal networks
o	Let computer identify patterns for plastic identification
	Look at two learning methods: spectral and SDCM
•	Spectral: use all information in the spectrum
•	SDCM: use correlations identified by computer
o	Proved to be superior to PCA Source
o	Feed patterns to the learning models
o	Use 6 different learning models


Struct:
Introdction to Application
Benefit of ML
Shortcomings of other classifier
Challenges for classification
Previous work
This work
