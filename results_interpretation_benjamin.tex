\subsection*{Interpretability of signatures and principal components}
Unsupervised dimensional reduction algorithms group high-dimensional data points
into a smaller number of clusters based on the similarity of features. This motivates
the idea of asigning a specific meaning or explanation to each cluster. 

We applied both SDCM and PCA to a subset of our data for which we had full
information of the categories \enquote{material type},\enquote{material
color},\enquote{material origin}, \enquote{Session ID} and enquote{Sample ID}.
\enquote{Session ID} identifies the session date and background spectrum,
\enquote{Sample ID} the unique sample the measurement was taken from.  The data
contained $1243$ measurements split into $3057$ spectral bins in the range of
\SIrange{410}{1002}{\nano\meter}  (long spectrum), and
\SIrange{410}{680}{\nano\meter} split into $1394$ spectral bins (short
spectrum). SDCM detected $143$ signatures for the long spectrum, and $121$
signatures for the short spectrum. As we did not cut off the selection of
principal components by variance, PCA terminated with $1243$ components for both
ranges.

SDCM and PCA provide signatures and principal components, meaning axes in high
dimensional feature space along which the data is correlated.  In the following
we refer to both SDCM signatures and principal components as \emph{signatures}.
We refer to a \emph{subsignature} as either the part of the signature below $0$
(the median), above $0$, or the entire signature together.  For both SDCM and
PCA, we assorted each measurement point in feature space a weight $w \in
\qty[0,1]$ quantifying its membership in the subsignature. We say a measurement
takes part in (or is contained by) a subsignature $\ssig{k}$ if $w \ge 0.05$.

Each measurement has a label containing information of the measurument
properties and material properties. $\infoCat$ is the set of categories from
which the labels are drawn. $\labelCol{\infoCat}$ is the set of labels induced by
the choice of category. The number of labels induced by various choices of
categories is shown in table \autoref{tab:nlabels}. 

Let $L \subset \LC$ be a set of labels. We are interested in subsignatures $\ssig{k}$
that describe $L$, i.e. a subsignature $\ssig{k}$ s.t. labels $l\in L$ significantly
take part in $\ssig{k}$, and labels $l \not\in L$ do not. For this we first
calculated the $p$ value for the contingency table of the label membership and
subsignature memebership via a Fisher-Exact test. Using the weights $w$, we
calculated the average weight in the subsignature of every label
$\aws{l}{\ssig{k}}$ and the membership score $\avmems{\ssig{k}}$ (see Methods for a
precise definition). $\avmems{\ssig{k}} \in \qty[0,1]$ quantifies how strongly
$\ssig{k}$ separates the labels $\LC$ into signature membership and
non-memberhsip ($\avmems{\ssig{k}} = 1$ means that all labels are either
completely contained or excldued by $\ssig{k}$). We say $\ssig{k}$ is
\emph{selecting} for $L$ if all $l \in L$ have $p<\tpval$ and
$\aws{l}{\ssig{k}} \ge \taw$, and $\avmems{\ssig{k}} > \tmem$. We say $\ssig{k}$
is \emph{singular} if it is selecting and $\abs{L} = 1$. 

In the following we choose $\tpval = 0.001$, $\taw = \tmem = 0.90$. \todo{maybe
some discussion on the thresholds?}
\begin{table}
    \centering
    \begin{tabular}{c | c | c}
        Category & Number of labels  & Number of labels with $m^l > 1$\\
        \hline
        Sample ID& 80&8\\ 
        Session ID& 88&0\\ 
        Type& 18&10\\ 
        Color& 13&11\\ 
        Origin & 5&5\\
        Type, Color, Origin & 57&19
    \end{tabular}
    \caption{Number of labels for several choices of categories}
    \label{tab:nlabels}
\end{table}
\subsubsection*{SDCM and PCA performance}
\autoref{fig:sdcmpcabeta} shows the distribution of $\avmems{\ssig{k}}$ for PCA
and SDCM over all subsignatures in both choices of spectral range for the finest
choice of categories (\emph{Session ID}). Whereas the
distrubution sharply peaks close to $1$ for SDCM, implying a large number
signatures that are selecting for some choice of $L$, the vast majority of PCA
subsignatures has a failry low value of $\avmems{ssig{k}}$. This is not
surprising, as we put not restrictions on the number of principal components,
and let the algorithm terminate at the number of measurements. A second, smaller
peak appears for PCA in the $0.9$ to $0.95$ range. Close to $1$, SDCM clearly
detects more selecting subsignatures in both absolute numbers and relative to
the overall number of subsignatures.

\autoref{fig:sdcmpcalabeldistribution} shows the distribution of labels taking
part in selecting subsignatures for SDCM and PCA. 

Decreasing the spectral range, shows a reduction of highly selecting
subsignatures in SDCM, and a general shift to lower $\avmems{\ssig{k}}$ in PCA.
This indicates that the tail of the spectra, although not containing interesting
features on visual inspection, can be relevant for detecting high dimensional
features in the measurements. \todo{Expand on this: Pictures of
spectra}\todo{Discuss spectra in methods?}\todo{are we sure there are no
artefacts Discuss security with 680nm mode}.
\begin{figure}
    \centering
    \includegraphics[width=\textwidth]{figures/hist_beta_distribution}
    \caption{Distribution of $\avmems{\ssig{k}}$ over all subsignatures for
    SDCM (left) and PCA (middle), and both choices of spectral range for the
    finest choice of categories (\emph{Session ID}).
    (right) shows a detail comparison of PCA and SDCM for the long spectral
    range.} 
    \label{fig:sdcmpcabeta}
\end{figure}
\begin{figure}
    \centering
    \includegraphics[width=\textwidth]{figures/hist_labelDistribution}
    \caption{Distribution of selecting signatures over the number of labels
    described for SDCM and PCA for the finest choice of categories
    (\emph{Session ID}).} 
    \label{fig:sdcmpcalabeldistribution}
\end{figure}


\subsubsection*{Signature generality}
We are interested in the ability of SDCM and PCA in containing describing broad
material properties categories such as \emph{type}, \emph{origin} or
\emph{color} with a single subsignature. We define the \emph{baseline
categories} $\infoCatt$ as the categories
\emph{type},\emph{origin},\emph{color}, \emph{Sample ID} and \emph{Session ID}.
Given a singular subsignature for label $l$, we define the \emph{label
multiplicity} $m^l$ as the number of unique labels in $\LCt$ that match $l$.
This step is required, as some general labels might only be represented once in
$\LCt$ (e.g. if a particular plastic type was only measured once on one sample).
Singular signatures with $m^l > 1$ can be said to generically describe $l$
\emph{within the data available}.\todo{maybe emphasise this point}

\autoref{tab:ssigTablegenerall} shows the number of unique labels in selecting
subsignatures, unique labels in singular subsignatures and unique labels in
singular subsignatures with $m^l > 1$ for both spectral ranges, several choices
of categories and PCA and SDCM. Although SDCM finds more overall lables from selecting and
singular subsignatures, both SDCM and PCA perform poorly in finding general
signatures for type, origin and color. Only for the cateogry \emph{type} SDCM
performs better than PCA. Comparing with the overall number of labels with $m^l
> 1$ in \autoref{tab:nlabels}, this indicates that neither method is well suited
to find univariate interpretations of the data on a broad scale. 
\begin{table}
    \centering
    \begin{tabular}{c | c || c | c | c | c}
        Spectral Range & Categories & Algorithm & \shortstack{Unique labels in
        \\ selecting Subsignatures} & \shortstack{Unique labels in \\ singular
        Subsignatures} & \shortstack{Unique labels in \\ Singular Subsignatures
        \\ with $m^l > 1$} \\
        \hline
        \SIrange{410}{1002}{\nano\meter} & Type, Origin, Color & PCA &15& 8& 1\\
                                         & & SDCM                    &29&15& 1\\
                                         & Type & PCA                & 0& 0& 0\\
                                         & & SDCM                    & 8& 4& 2\\
                                         & Origin & PCA              & 0& 0& 0\\
                                         & & SDCM                    & 0& 0& 0\\
                                         & Color & PCA               & 1& 1& 0\\
                                         & & SDCM                    & 2& 2& 0\\
        \SIrange{410}{680}{\nano\meter} & Type, Origin, Color & PCA  &14& 8& 1\\
                                         & & SDCM                    &23&10& 1\\
                                         & Type & PCA                & 0& 0& 0\\
                                         & & SDCM                    & 9& 2& 1\\
                                         & Origin & PCA              & 0& 0& 0\\
                                         & & SDCM                    & 0& 0& 0\\
                                         & Color & PCA               & 1& 1& 0\\
                                         & & SDCM                    & 2& 2& 0
    \end{tabular}
    \caption{Overview of numbers of labels detected for
    material property labels. }
    \label{tab:ssigTablegenerall}
\end{table}
\subsubsection*{Signature specificity}
The finest partition of the measurements is given by the categories \emph{Sample
ID} and \emph{Session ID}. Here, SDCM and PCA find the comparable numbers of
unique labels in selecting subsignatures. However, SDCM clearly outperforms PCA
in finding singular subsignatures. We note that this contrast gets more extreme,
the higher the thresholds $\taw$ and $\tmem$ are chosen \todo{compare figure? or
another table}.  SDCM was also able to group together measurements of the sample
ID for different measurement sessions.

\autoref{tab:listUniqSampleID} shows the material info of those $33$ samples which
can be uniquely identified by SDCM. Only one non-plastic can be identified at
these threshold values, while the majority is made up of supermarket $PE$, $PET$
and $PP$ samples.

These results indicate that SDCM is capable of detecting high-dimensional
similarities between spectra down to the level of discerning different samples
in commercial plastics.  This is relevant for further applications in
environmental science, such as identifying single sources of microplastic
contamination from a broad range of measurements, and tracing contaminations
back towards their sources either spatially or in time.

Further work is required to test the extent and reliability of this method. So
far, only $33$ of $80$ samples could be uniquely identified by SDCM. Only in $2$
samples measurements from multiple sessions (and thus possible differing
experimental conditions) were present. Extending the available data, and
performing
measurements of the same sample in multiple sessions under varying experimental
conditions is necessary to strengthen this method. We also do not provide a
method for determining when a subsignature describes a single sample for novel
measurements with unknown labeling.\todo{difficulty of enumerating all possible
combinations}

\begin{table}
    \centering
    \begin{tabular}{c | c || c | c | c | c}
        Spectral Range & Categories & Algorithm & \shortstack{Unique labels in
        \\ selecting Subsignatures} & \shortstack{Unique labels in \\ singular
        Subsignatures} & \shortstack{Unique labels in \\ Singular Subsignatures
        \\ with $m^l > 1$} \\
        \hline
        \SIrange{410}{1002}{\nano\meter} & Sample ID & PCA             &79&15& 0\\
                                         & & SDCM                      &72&33& 2\\
                                         & Session ID & PCA            &87&16& 0\\
                                         & & SDCM                      &81&36& 0\\
                                        \hline
        \SIrange{410}{680}{\nano\meter}  & Sample ID & PCA             &70&11& 0\\
                                         & & SDCM                      &74&23& 1\\
                                         & Session ID & PCA            &80&14& 0\\
                                         & & SDCM                      &82&27& 0
    \end{tabular}
    \caption{Overview of numbers of labels detected for
    specific sample and session labels. }
    \label{tab:ssigTablespecific}
\end{table}
\begin{table}
    \centering
    \begin{tabular}{c|c|c}
        Type & Origin & Color \\
        %Sample ID & Type & Origin & Color \\
        \hline
        %NpNat0001M00    & N1  &    Elba       &    cream \\
        %PlPcBpi0000M00  & PC  &    BPI        &    clear \\
        %PlPcBre0000M00  & PC  &    Brett      &    clear \\
        %PlHdpeRet0096M00& PE  &    supermarket&    blue  \\
        %PlLdpeRet0023M00& PE  &    supermarket&    green \\
        %PlHdpeRet0046M00& PE  &    supermarket&    grey  \\
        %PlHdpeRet0125M00& PE  &    supermarket&    pink  \\
        %PlHdpeRet0010M00& PE  &    supermarket&    white \\
        %PlLdpeRet0169M00& PE  &    supermarket&    yellow\\
        %PlPetRet0042M00 & PET &    supermarket&    black \\
        %PlPetRet0071M00 & PET &    supermarket&    black \\
        %PlPetRet0081M00 & PET &    supermarket&    blue  \\
        %PlPetRet0113M00 & PET &    supermarket&    green \\
        %PlPetRet0126M00 & PET &    supermarket&    pink  \\
        %PlPetRet0153M00 & PET &    supermarket&    purple\\
        %PlPetRet0084M00 & PET &    supermarket&    red   \\
        %PlPetRet0035M00 & PET &    supermarket&    yellow\\
        %PlPetRet0156M00 & PET &    supermarket&    yellow\\
        %PlPmmaBre0000M00& PMMA&    Brett      &    clear \\
        %PlPpRet0043M00  & PP  &    supermarket&    black \\
        %PlPpRet0037M00  & PP  &    supermarket&    blue  \\
        %PlPpRet0097M00  & PP  &    supermarket&    blue  \\
        %PlPpRet0075M00  & PP  &    supermarket&    green \\
        %PlPpRet0086M00  & PP  &    supermarket&    green \\
        %PlPpRet0163M00  & PP  &    supermarket&    red   \\
        %PlPpRet0164M00  & PP  &    supermarket&    red   \\
        %PlPpRet0014M00  & PP  &    supermarket&    white \\
        %PlPpRet0019M00  & PP  &    supermarket&    yellow\\
        %PlPpRet124bM00  & PP  &    supermarket&    yellow\\
        %PlPsBam0000M00  & PS  &    BAM        &    clear \\
        %PlPsBpi0000M00  & PS  &    BPI        &    clear \\
        %PlPsBre0000M00  & PS  &    Brett      &    clear \\
        %PlPvcBam0000M00 & PVC &    BAM        &    white 

        N1  &    Elba       &    cream \\
        PC  &    BPI        &    clear \\
        PC  &    Brett      &    clear \\
        PE  &    supermarket&    blue  \\
        PE  &    supermarket&    green \\
        PE  &    supermarket&    grey  \\
        PE  &    supermarket&    pink  \\
        PE  &    supermarket&    white \\
        PE  &    supermarket&    yellow\\
        PET &    supermarket&    black \\
        PET &    supermarket&    black \\
        PET &    supermarket&    blue  \\
        PET &    supermarket&    green \\
        PET &    supermarket&    pink  \\
        PET &    supermarket&    purple\\
        PET &    supermarket&    red   \\
        PET &    supermarket&    yellow\\
        PET &    supermarket&    yellow\\
        PMMA&    Brett      &    clear \\
        PP  &    supermarket&    black \\
        PP  &    supermarket&    blue  \\
        PP  &    supermarket&    blue  \\
        PP  &    supermarket&    green \\
        PP  &    supermarket&    green \\
        PP  &    supermarket&    red   \\
        PP  &    supermarket&    red   \\
        PP  &    supermarket&    white \\
        PP  &    supermarket&    yellow\\
        PP  &    supermarket&    yellow\\
        PS  &    BAM        &    clear \\
        PS  &    BPI        &    clear \\
        PS  &    Brett      &    clear \\
        PVC &    BAM        &    white 
    \end{tabular}
    \caption{Listing of the $33$ Sample IDs uniqueley selected by SDCM
    subsignatures.}
    \label{tab:listUniqSampleID}
\end{table}

%\autoref{tab:selectingTotals} shows total numbers of selecting subsignatures for
%SDCM and two methods of calculating the PCA signature membership \todo{remove
%soft weights?}\todo{show mixing signatures?}. For all three choices of
%categories, SDCM detects the highest number of selecting subsignatures.
%
%\autoref{fig:histsSpecSig} shows the distribution of selecting subsignatures over
%the amount of labels contained in each explanation. For the finest choice of categories
%\autoref{sfig:histSpecSig1}, PCA (hard) detects $33$ selecting signatures, each
%containing up to $26$ labels\todo{check these numbers}, but very few signatures
%selecting only one or two labels. SDCM conversely only detects signatures
%selecting only two or fewer labels. This demonstrates that SDCM can detect finer
%differences between datasets, singling out measurements that have been performed
%on the same material sample.\todo{Discussion -> problem large hetergoeneities?}.
%The labels selected by SDCM are displayed in \todo{table einfügen}. 
%
%Coarsening the categories
%(\autoref{sfig:histSpecSig2},\autoref{sfig:histSpecSig3})reduces the number of
%selecting subsignatures. This is expected, as mixing finer lables into a coarser
%set combines heterogeneities into a single label, which were previously represented
%on separate signatures. For the coarsest set of categories ("Type"), only SDCM is
%able to find selecting signatures.
%\begin{figure}[h]     
%    \begin{subfigure}[b]{0.3\textwidth}
%        \centering
%        \includegraphics[width=\textwidth]{figures/hist_typeOriginColorId1.png}
%        \caption{Type/Origin/Color/Sample ID}
%        \label{sfig:histSpecSig1}
%    \end{subfigure}
%    \begin{subfigure}[b]{0.3\textwidth}
%        \centering
%        \includegraphics[width=\textwidth]{figures/hist_typeOriginColor.png}
%        \caption{Type/Origin/Color}
%        \label{sfig:histSpecSig2}
%    \end{subfigure}
%    \begin{subfigure}[b]{0.3\textwidth}
%        \centering
%        \includegraphics[width=\textwidth]{figures/hist_type.png}
%        \caption{Type}
%        \label{sfig:histSpecSig3}
%    \end{subfigure}
%    \caption{Distribution of the number of selecting subsignatures for three
%    choices of categories.} 
%    \label{fig:histsSpecSig}
%\end{figure}
