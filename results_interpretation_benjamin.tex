\subsection*{Interpretability of principal components signatures}
Unsupervised dimensional reduction algorithms group high-dimensional data points
into a smaller number of clusters based on the similarity of features. This motivates
the idea of asigning a specific meaning or explanation to each cluster. 

A \emph{specific subsignature} contains only mesaurements of a subset of labels,
contains nearly all representatives of that subset and almost none of any other
labels \todo{too unclear?}. Thus we can assign that particular subset of labels
as an interpretation of that signature. 
\todo{redo images with new selection rule}

\begin{figure}[h]     
    \begin{subfigure}[b]{0.3\textwidth}
        \centering
        \includegraphics[width=\textwidth]{figures/hist_typeOriginColorId1.png}
        \caption{Type/Origin/Color/Sample ID}
        \label{sfig:histSpecSig1}
    \end{subfigure}
    \begin{subfigure}[b]{0.3\textwidth}
        \centering
        \includegraphics[width=\textwidth]{figures/hist_typeOriginColor.png}
        \caption{Type/Origin/Color}
        \label{sfig:histSpecSig2}
    \end{subfigure}
    \begin{subfigure}[b]{0.3\textwidth}
        \centering
        \includegraphics[width=\textwidth]{figures/hist_type.png}
        \caption{Type}
        \label{sfig:histSpecSig3}
    \end{subfigure}
    \caption{Distribution of the number of selecting subsignatures for three
    choices of categories.} 
    \label{fig:histsSpecSig}
\end{figure}
\begin{table}
    \centering
    \begin{tabular}{c|c|c|c|c}
        Categories & Weight Name& Significant & Mixing & Selecting   \\ 
        \hline
        \hline
        Type/Origin/Color/Sample ID & SDCM Weights         &   28.904   &    27.972  &   8.8578\\  
                                    & PCA Weights (soft)   &   29.869   &    2.0311  &  0.11947 \\ 
                                    & PCA Weights (hard)   &   33.094   &    11.111  &  2.7479  \\
        \hline
        Type/Origin/Color& SDCM Weights         &   27.273   &    22.145  &  3.9627  \\
                         & PCA Weights (soft)   &   30.585   &    1.3142  &  0.11947 \\
                         & PCA Weights (hard)   &   33.214   &    7.6464  &  0.59737 \\
        \hline
        Type & SDCM Weights         &   21.678   &    13.986  &   0.6993 \\ 
             & PCA Weights (soft)   &   22.461   &         0  &        0 \\ 
             & PCA Weights (hard)   &   23.178   &    1.7921  &        0 \\
    \end{tabular}
    \caption{Percentage of subsignatures that are Significant, Mixing or
    Selecting for three different choices of categories.}
    \label{tab:selectingPercentages}
\end{table}
\begin{table}
    \centering
    \begin{tabular}{c|c|c|c|c}
        Categories & Weight Name& Significant & Mixing & Selecting   \\ 
        \hline
        \hline
        Type/Origin/Color/Sample ID& SDCM Weights         &   124   &  120   & 38\\  
                                   & PCA Weights (soft)   &   250   &    17  &  1 \\ 
                                   & PCA Weights (hard)   &   277   &    93  &  23  \\
        \hline
        Type/Origin/Color& SDCM Weights         &   117   &    95  &  17  \\
                         & PCA Weights (soft)   &   256   &    11  &  1 \\
                         & PCA Weights (hard)   &   278   &    64  &  5 \\
        \hline
        Type & SDCM Weights         &   93    &    60  &        3 \\ 
             & PCA Weights (soft)   &   188   &     0  &        0 \\ 
             & PCA Weights (hard)   &   194   &    15  &        0 \\
    \end{tabular}
    \caption{Total number of subsignatures that are Significant, Mixing or
    Selecting for three different choices of categories.}
    \label{tab:selectingTotals}
\end{table}
\autoref{tab:selectingPercentages} and \autoref{tab:selectingTotals} show the
percentages and total numbers of the significant, mixing and selecting
subsignatures\todo{Let PCA run through all PCs?}. For all three choices of
categories, SDCM achieves the highest percentage and absolute number of both
mixing and selecting subsignatures.

\autoref{fig:histsSpecSig} shows the distribution of specfic subsignatures over
the amount of labels contained in each explanation.\todo{Maybe change this to
normalized plots?}. For the finest choice of categories
\autoref{sfig:histSpecSig1}, PCA (hard) detects $23$ selecting signatures, each
containing up to $26$ labels, but very few signatures selecting only one or two
labels. SDCM conversly only detects signatures selecting only two or fewer
labels. This demonstrates that SDCM can detect finer differences between
datasets, singling out measurements that have been performed on the same material
sample.\todo{Discussion -> problem large hetergoeneities?}. The labels selected
by SDCM are displayed in \todo{table einfügen}. 

Coarsening the categories
(\autoref{fig:histSpecSig2},\autoref{fig:histSpecSig3})reduces the number of
selecting subsignatures. This is expected, as mixing finer lables into a coarser
set combines heterogeneities into a single label, which were previously represented
on separe signatures. For the coarsest set of categories ("Type"), only SDCM is
able to find selecting signatures.
