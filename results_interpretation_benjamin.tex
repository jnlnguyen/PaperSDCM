\subsection*{Interpretability of signatures and principal components}
\subsubsection*{Matching signatures to labels}
\todo{redo images with new selection rule}
Unsupervised dimensional reduction algorithms group high-dimensional data points
into a smaller number of clusters based on the similarity of features. This motivates
the idea of asigning a specific meaning or explanation to each cluster. 

Each measurement has a label containing information of the measurument and
material properties. $\infoCat$ is the set of categories from which we draw our
labels. $\labelCol{\infoCat}$ is the set of labels induced by the choice of
category.  Let $L \in \LC$ be a set of labels. 

We applied both SDCM and PCA to a subset of our data for which we had full
information of the categories \enquote{material type},\enquote{material
color},\enquote{material origin} and enquote{sample ID}. \enquote{sample ID} is
a unique identifier for the sample the measurement was take from. The data
contained $1243$ measurements in the range of
\SIrange{410}{1002}{\nano\meter}, split into $3057$ spectral bins.
SDCM detected $143$ signatures. As we did not cut off the selection of principal
components by variance, PCA terminated with $1243$ components.

In the following we refer to both SDCM signatures and principal components as
\emph{signatures}. We refer to a \emph{subsignature} as either the part of the
signature below above $0$ (the median), above $0$, or the entire signature together.
%A subsignature is \emph{mixing} for $L$ if it contains almost all measurements
%of this set of labels. It is furthermore \emph{selecting} for $L$ if it contains
%almost no measurements with any other labels $l' \not\in L$ (see the mothods for
%a precise definition of these terms).
A subsignature contains a measurement, if the measurement has weight $w>0.05$
for that subsignature. A subsignature is \emph{selecting} for $L$ if it contains
almost all measurements with labels $l \in L$, and almost no measurements with
any other labels $l' \not\in L$ (see the mothods for a precise definition of
these terms).

\autoref{tab:selectingTotals} shows total numbers of selecting subsignatures for
SDCM and two methods of calculating the PCA signature membership \todo{remove
soft weights?}\todo{show mixing signatures?}. For all three choices of
categories, SDCM detects the highest number of selecting subsignatures.

\autoref{fig:histsSpecSig} shows the distribution of selecting subsignatures over
the amount of labels contained in each explanation. For the finest choice of categories
\autoref{sfig:histSpecSig1}, PCA (hard) detects $33$ selecting signatures, each
containing up to $26$ labels\todo{check these numbers}, but very few signatures
selecting only one or two labels. SDCM conversely only detects signatures
selecting only two or fewer labels. This demonstrates that SDCM can detect finer
differences between datasets, singling out measurements that have been performed
on the same material sample.\todo{Discussion -> problem large hetergoeneities?}.
The labels selected by SDCM are displayed in \todo{table einfügen}. 

Coarsening the categories
(\autoref{sfig:histSpecSig2},\autoref{sfig:histSpecSig3})reduces the number of
selecting subsignatures. This is expected, as mixing finer lables into a coarser
set combines heterogeneities into a single label, which were previously represented
on separate signatures. For the coarsest set of categories ("Type"), only SDCM is
able to find selecting signatures.
\begin{figure}[h]     
    \begin{subfigure}[b]{0.3\textwidth}
        \centering
        \includegraphics[width=\textwidth]{figures/hist_typeOriginColorId1.png}
        \caption{Type/Origin/Color/Sample ID}
        \label{sfig:histSpecSig1}
    \end{subfigure}
    \begin{subfigure}[b]{0.3\textwidth}
        \centering
        \includegraphics[width=\textwidth]{figures/hist_typeOriginColor.png}
        \caption{Type/Origin/Color}
        \label{sfig:histSpecSig2}
    \end{subfigure}
    \begin{subfigure}[b]{0.3\textwidth}
        \centering
        \includegraphics[width=\textwidth]{figures/hist_type.png}
        \caption{Type}
        \label{sfig:histSpecSig3}
    \end{subfigure}
    \caption{Distribution of the number of selecting subsignatures for three
    choices of categories.} 
    \label{fig:histsSpecSig}
\end{figure}
%\begin{table}
%    \centering
%    %\begin{tabular}{c|c|c|c|c}
%    %    Categories & Weight Name& Significant & Mixing & Selecting   \\ 
%    %    \hline
%    %    \hline
%    %    Type/Origin/Color/Sample ID & SDCM Weights         &   28.904   &    27.972  &   8.8578\\  
%    %                                & PCA Weights (soft)   &   29.869   &    2.0311  &  0.11947 \\ 
%    %                                & PCA Weights (hard)   &   33.094   &    11.111  &  2.7479  \\
%    %    \hline
%    %    Type/Origin/Color& SDCM Weights         &   27.273   &    22.145  &  3.9627  \\
%    %                     & PCA Weights (soft)   &   30.585   &    1.3142  &  0.11947 \\
%    %                     & PCA Weights (hard)   &   33.214   &    7.6464  &  0.59737 \\
%    %    \hline
%    %    Type & SDCM Weights         &   21.678   &    13.986  &   0.6993 \\ 
%    %         & PCA Weights (soft)   &   22.461   &         0  &        0 \\ 
%    %         & PCA Weights (hard)   &   23.178   &    1.7921  &        0 \\
%    %\end{tabular}
%    \caption{Percentage of subsignatures that are Significant, Mixing or
%    Selecting for three different choices of categories.}
%    \label{tab:selectingPercentages}
%\end{table}
\begin{table}
    \centering
    \begin{tabular}{c|c|c}
        Categories & Weight Name&  Selecting   \\ 
        \hline
        \hline
        Type/Origin/Color/Sample ID & SDCM Weights            &   45  \\  
                                    & PCA Weights (soft)      &    1  \\ 
                                    & PCA Weights (hard)      &   33  \\
        \hline                                                     
        Type/Origin/Color           & SDCM Weights            &   19  \\
                                    & PCA Weights (soft)      &    1  \\
                                    & PCA Weights (hard)      &    9  \\
        \hline                                                      
        Type                        & SDCM Weights            &    3   \\ 
                                    & PCA Weights (soft)      &    0   \\ 
                                    & PCA Weights (hard)      &    0   \\
    \end{tabular}
    \caption{Total number of selecting subsignatures for three different choices of categories.}
    \label{tab:selectingTotals}
\end{table}
% mixing
%&   238
%&    23
%&   775
%
%&   188
%&    17
%&   607
%       
%&    83
%&     0
%&   137    %                      Significant    Mixing    Specific

    %SDCM Weights              336         238         45   
    %PCA Weights (soft)       1635          23          1   
    %PCA Weights (hard)       1562         775         33   

    %SDCM Weights              329         188         19   
    %PCA Weights (soft)       1553          17          1   
    %PCA Weights (hard)       1741         607          9
    %
    %SDCM Weights              253          83         3    
    %PCA Weights (soft)        559           0         0    
    %PCA Weights (hard)        757         137         0  
\subsubsection*{Interpreting signature axes}
