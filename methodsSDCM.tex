    To model the influence of above heterogeneities on the observable spectra,
    we adopt a superposition model  
    \begin{equation*}
        S = \sum_{k=1}^n E_k + \eta
    \end{equation*}
    where $S$ denotes the $n_F\times n_M$ matrix of $n_F$ features and $n_M$
    measurements, and $\eta$ the residual noise of the measurements.

    To reverse the sum operator in this superposition and gain signatures, and
    ideally understanding, of heterogeneities that allow discerning the plastic
    types, we have applied SDCM, a signal dissection method originally developed
    for gene expression of cancer cells that are also complex superpositions of
    many simultaneously measured interactions or technical characteristics of
    the respective measurement pipeline. 

    Additionally, SDCM has been shown to
    dissect the matrix sum of simulated effects in a highly specific way, i.e.
    signatures correspond almost 1:1 with simulated effects, whereas other
    methods mixed simulated effects, such as PCA due to orthogonality
    constraints. As we aim not only to predict plastics type with high accuracy,
    but ideally also identify signatures that are explainable or univariately
    highly specific to certain physical plastic properties, we selected SDCM for
    dimension reduction.

    First, we preprocessed and normalized our data to remove known artefacts and
    known measurement biases; our preprocessing and normalization pipeline is
    explained in detail in \autoref{sub:Data preparation pipeline}

    Then, we randomly shuffled our n samples in a replicate(????, eigentlich
    type)-aware way to set a uniformly subselected 20\% partition aside for
    later validation. The remaining two thirds of samples were used to discover
    signatures, and both train and optimize predictor models based on the SDCM
    projections on these signatures. To keep the data consistent, the 80\% data
    used for the dissection was re-projected on the discovered axes.

    An analysis flow diagram is shown in (TODO).
