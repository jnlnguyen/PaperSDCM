\subsection*{Performance evaluation of ML models}
The prediction performance of a ML model depends on the intended application, the data transformation algorithm and the learning method.
Therefore, we first compare all ML models that are generated with all possible combinations of transformation algorithm and learning method.
For our evaluation, we calculate the performance metrics \textit{accuracy}, \textit{precision}, \textit{recall} and \textit{f1}.
\autoref{fig:metricComparison} gives an overview of the metrics for all generated models.
We immediately see that the learning method has the biggest influence on the model's performance.
Models generated with the SVC, NuSVC, Logistic Regression and the Random Forest
algorithms achieve values over \SI{90}{\percent} for all metrics while it drops by
\SI{###}{\percent} for models generated with the naive Bayes method.

The influence of the transformation algorithm, on the other hand, depends on the selected learning method.
We see similar values for models generated with the NuSVC and the logistic regression method, while for the rest, a data dimension reduction improves the performance by at least \SI{20}{\percent}.


\begin{figure}[ht]
	\centering
	\includegraphics[width=\textwidth]{figures/metric_evaluation.png}
	\caption{Comparison of classification results for different metrics. The
		black bars represent the standard deviations of all validations. As the
		Naive Bayes model scores significantly worse in all metrics than the other
		classifiers, it has been omitted from the image. Similarly, the input method
		PCA (no batchprocessing) has been omitted, as it scores equivalenlty to PCA
		on all metrics.}
	\label{fig:metricComparison}
\end{figure}
\subsection*{Classification performance of ML models}

\begin{figure}[ht]
	\centering
	\includegraphics[width=\textwidth]{figures/confusionMatrix_validation.png}
	\caption{Confusion matrix of the validation set for individual sample
		classes and classifiers, normalized along rows. The heatmaps are to be read
		as \enquote{in <p>\%
			of all predictions <row> is classified as <column>}. As the Naive Bayes
		model scores significantly worse in all metrics than the other classifiers,
		it has been omitted from the image.}
	\label{fig:confusionMatrix}
\end{figure}

Next, we evaluate the performance of our models with respect to the different sample types in our dataset.
Figure~\ref{fig:confusionMatrix} presents a confusion matrix of all high-scoring models in this study.

For both spectral input methods, there is a signficant confusion of PVC and
nonplastics. These classes are known to have very similar spectra \todo{image?}.

The image reveals that the performance for an individual class depends on the dimension reduction technique and learning method.
For example, a model that uses random forest and SDCM-transformed data is better at identifying PA than a random forest model with PCA-transformed data.
We also observe trends that are present in all models: first, PP gets mixed up as PE and second, PVC gets mixed up as a non-plastic material.
These observations show, that more data is required so that the dimension
reduction techniques can capture the spectral signatures to identify the
classes.\todo{This paragraph needs some work}

Performance values of ML models:
-	Kommentar zu overfitting
-	Conclusion erst nach präsentation der daten
-	Kommentar zum Ergebniss, dass SDCM für lineare Methoden besser ist.
o	Was impliziert das? Wann sind lin. Methoden besser? Sind lin methoden für unser problem besser?
-	„Dimension red. technique…has little influence on the performance“ : Aussage stimmt nicht mehr wenn man sich spectra anschaut. Ist Übergang von spectra auf PCA und SDCM nicht logisch wegen curse of dimensionality?
-	Aussage:
o	Alle metriken über  90
	Sdcm is besser bei linear
	Pca besser bei nicht linear
Figure 1:
-	NuSvc rauslassen
-	Precision und Recall wichtig wenn f1 genutzt wird?
-	Naive Bayes rauslassen
Classification performance of ML models:
-	PCA and SDCM modelle sind vergleichbar
-	PCA and SDCM erhöhen die PVC Detektion.
o	Wie ist die PVC detektion bei anderen Techniken?

