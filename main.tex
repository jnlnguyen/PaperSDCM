\documentclass[fleqn,10pt]{wlscirep}
\usepackage[utf8]{inputenc}
\usepackage[T1]{fontenc}
\title{Scientific Reports Title to see here}

\author[1,*]{Alice Author}
\author[2]{Bob Author}
\author[1,2,+]{Christine Author}
\author[2,+]{Derek Author}
\affil[1]{Affiliation, department, city, postcode, country}
\affil[2]{Affiliation, department, city, postcode, country}

\affil[*]{corresponding.author@email.example}

\affil[+]{these authors contributed equally to this work}

%\keywords{Keyword1, Keyword2, Keyword3}

\begin{abstract}
    Microplastic contamination is an increasingly problematic risk for the
    environment and human health. To quantify its impact it is necessary to
    develop practical and cost-efficient of measuring and identifying
    microplastic samples. Here we present the classfication of microplastics
    based on photoluminesence spectra taken with an inexpensive set up, and
    pre-existing classification algorithms. For dimensional reduction we also
    use SDCM, a novel algorithm for finding corrleations in high-dimensional
    data.
\end{abstract}
\begin{document}

\flushbottom
\maketitle
% * <john.hammersley@gmail.com> 2015-02-09T12:07:31.197Z:
%
%  Click the title above to edit the author information and abstract
%
\thispagestyle{empty}

\noindent Please note: Abbreviations should be introduced at the first mention in the main text – no abbreviations lists. Suggested structure of main text (not enforced) is provided below.

\section*{Introduction}
    Microplastic contamination is an increasingly problematic risk for the
    environment, and in the end potentially also for human health, if these
    particles get into the nutrition chain. To identify causes and remedies it
    is essential to quantify which type of plastic material the microplastic
    contaminations are made of. A cost effective and scalable approach is
    photoluminescence. Here, we investigate if these light spectra are
    sufficient for robust predictions of plastic type and other sample
    characteristics.

\section*{Results}

\subsection*{Experiments}
\subsection*{Classification}
    \begin{table}[ht]
        \centering
        \begin{tabular}{c c||c|c|c|c}
            \hline
            Input data & Spectral range & Set &Accuracy & Precision & Recall \\
            \hline
            Spectra & 410nm--860nm  & Test&1 & 1 & 1  \\
                    & 410nm--860nm  & Validation&1 & 1 & 1  \\
                    & 410nm--1010nm & Test&1 & 1 & 1  \\
                    & 410nm--1010nm & Validation&1 & 1 & 1  \\
            PCA     & 410nm--860nm  & Test&1 & 1 & 1   \\
                    & 410nm--860nm  & Validation&1 & 1 & 1   \\
                    & 410nm--1010nm & Test&1 & 1 & 1   \\
                    & 410nm--1010nm & Validation&1 & 1 & 1   \\
            SDCM    & 410nm--860nm  & Test&1 & 1 & 1  \\
                    & 410nm--860nm  & Validation&1 & 1 & 1  \\
                    & 410nm--1010nm & Test&1 & 1 & 1 \\  
                    & 410nm--1010nm & Validation&1 & 1 & 1  
        \end{tabular}
        \caption{Classification results.}
        \label{tab:dataoverview}
    \end{table}
\subsection*{Signature analysis}
\section*{Discussion}


\section*{Methods}
\subsection*{Measurements}
\subsection*{Classifciation}
\subsubsection*{Data composition}
    We have measured X samples of Y different plastic types, for several
    overlapping characteristics, both known, like color, and unknown in the pipeline
    of the respective manufacturer.
    \begin{table}[ht]
        \centering
            \begin{tabular}{c||c|c}
                \hline
                Set & Percentage of total data & \# Measurements  \\
                \hline
                Training+Testing & 80\%    & 1000        \\
                Validation & 20\% & 200 
            \end{tabular}
        \caption{Listing of the used samples and their properties.}
        \label{tab:setoverview}
    \end{table}
    \begin{table}[ht]
        \centering
            \begin{tabular}{c|c|c|c|c|c}
                \hline
                Sample Type & Color & Origin & Sample Form & \# Measurements &
                \# Replicates\\
                \hline
                PE & red    & BAM       & Bulk & 9999 & 2 \\
                   & blue   & BAM       & Bulk & 9999 & 2 \\
                   & blue   & unknown   & Bulk & 9999 & 2 \\
                PMMA & red & BAM & Bulk & 9999 & 2 \\
            \end{tabular}
        \caption{Listing of the used samples and their properties.}
        \label{tab:dataoverview}
    \end{table}
\subsection*{Data preparation}
\subsection*{SDCM}
    To model the influence of above heterogeneities on the observable spectra,
    we adopt a superposition model  
    \begin{equation*}
        S = \sum_{k=1}^n E_k + \eta
    \end{equation*}
    where $S$ denotes the $n_F\times n_M$ matrix of $n_F$ features and $n_M$
    measurements, and $\eta$ the residual noise of the measurements.

    To reverse the sum operator in this superposition and gain signatures, and
    ideally understanding, of heterogeneities that allow discerning the plastic
    types, we have applied SDCM, a signal dissection method originally developed
    for gene expression of cancer cells that are also complex superpositions of
    many simultaneously measured interactions or technical characteristics of
    the respective measurement pipeline. 

    Additionally, SDCM has been shown to
    dissect the matrix sum of simulated effects in a highly specific way, i.e.
    signatures correspond almost 1:1 with simulated effects, whereas other
    methods mixed simulated effects, such as PCA due to orthogonality
    constraints. As we aim not only to predict plastics type with high accuracy,
    but ideally also identify signatures that are explainable or univariately
    highly specific to certain physical plastic properties, we selected SDCM for
    dimension reduction.

    First, we preprocessed and normalized our data to remove known artefacts and
    known measurement biases; our preprocessing and normalization pipeline is
    explained in detail in \autoref{sub:Data preparation pipeline}

    Then, we randomly shuffled our n samples in a replicate(????, eigentlich
    type)-aware way to set a uniformly subselected 20\% partition aside for
    later validation. The remaining two thirds of samples were used to discover
    signatures, and both train and optimize predictor models based on the SDCM
    projections on these signatures. To keep the data consistent, the 80\% data
    used for the dissection was re-projected on the discovered axes.

    An analysis flow diagram is shown in (TODO).
    \subsection*{Classification}
    The classification was performed with the standard tools provided by the
    PYTHON sklearn library. The

\bibliography{sample}

\section*{Acknowledgements}
\section*{Additional information}
\subsection*{Data preparation pipeline}
\label{sub:Data preparation pipeline}
    Before the classfication, the data was prepared with several steps:
    \subsubsection*{Interpolation} 
        The different spectra were interpolated onto a common axis. Here the
        number of interpolation points wer
    \subsubsection*{Background Subtraction} 
        For each batch measurement a background measurement was recorded. Each
        spectrum was subtracted by the background measurement.
    \subsubsection*{Filtering} 
    \subsubsection*{Featuredesign} 
\subsection*{Classification pipeline}

%To include, in this order: \textbf{Accession codes} (where applicable); \textbf{Competing interests} (mandatory statement). 
%
%The corresponding author is responsible for submitting a \href{http://www.nature.com/srep/policies/index.html#competing}{competing interests statement} on behalf of all authors of the paper. This statement must be included in the submitted article file.

%\begin{figure}[ht]
%\centering
%\includegraphics[width=\linewidth]{figures/stream}
%\caption{Legend (350 words max). Example legend text.}
%\label{fig:stream}
%\end{figure}
%
%\begin{table}[ht]
%\centering
%\begin{tabular}{|l|l|l|}
%\hline
%Condition & n & p \\
%\hline
%A & 5 & 0.1 \\
%\hline
%B & 10 & 0.01 \\
%\hline
%\end{tabular}
%\caption{\label{tab:example}Legend (350 words max). Example legend text.}
%\end{table}
%
%Figures and tables can be referenced in LaTeX using the ref command, e.g. Figure \ref{fig:stream} and Table \ref{tab:example}.

\end{document}
