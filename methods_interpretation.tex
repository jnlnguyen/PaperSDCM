\subsection*{Physical interpretation of signatures}
To generate physically interpretable signatures from the dataset, we restricted
the measurements to those for which the properties type, origin, color,
isPlastic and sampleID are fully known. This reduced the input matrix to
size $n\times m $\todo{insert values}. 

In the following we define \emph{subsignatures} $k^*$ of a signature $k$ as
those measurements which are consistently expessed stronger than the median
along the signature axes in spectral space ($k^+$), consistently expressed
weaker than the median ($k^-$), or all measurements ($k^0$).

SDCM readily provides weights $w$ which quantify how much a measurement
participates in a $k$. As the implementations of PCA in
matlab and sklearn \todo{add ref or fancy this up} do no provide a comparable
metric, we need to define the PCA axis weight for comparison.
Let $\cpca{k}{j}$ be the coefficient of the $k$th principal component of the
$j$th measruement. We then define 

\begin{align*}
    \wpcas{k}{j}
        &= 
        \sign\qty(\cpca{k}{j})
        \min
        \pqty{
                1,  
                \frac{
                    \abs{\cpca{k}{j}}
                }
                {
                    0.5 \cdot \max_{j\in\mSet}\abs{\cpca{k}{j}}
                }
            }_j \\
    \wpcah{k}{j}
        &= 
        \begin{cases}
            \sign\qty(\cpca{k}{j}) & \text{if }
            \frac{\abs{\cpca{k}{j}}}{\max_{j\in\mSet} \abs{\cpca{k}{j}}} > 0.05  \\  
            0 &  \text{else.}
        \end{cases}
\end{align*}

A measurement $j$ is considered to be part of a subsignature ${k^*}$, if
$\abs{\weight{k^*}{j}} \ge 0.05$.  Using the binary variables \enquote{is part
of subsignature $\ssig{k}$} and \enquote{is part of label $l$}, we constructed a
contingency table for each subsignature-label combination, and calculate a $p$
value using a Fisher exact test.  \todo{hypgergeometric test?}\todo{cite
implementation}. We used the Benjamini-Hochberg procedure to calculate the falls
discover rate (FDR).\todo{citaitons} 

The dataset separates into $n$\todo{insert value} subclasses, which are listed
in \todo{insert ref and citation}.
For each subsignature ${k^*}$ and subclass $l$ we calculate the following
metrics
\begin{equation*}
    \aws{l}{{k^*}} = \frac{1}{N_l}\sum_j \abs{w^{k^*}_j} \in [0,1], \qquad
    \mems{l}{{k^*}} = \abs{2\cdot\aws{l}{{k^*}}-1} \in [0,1], \qquad
    \avmems{{k^*}} = \frac{\sum_l N_l \mems{l}{{k^*}}}{\sum_l N_l} \in [0,1].
\end{equation*}
$\aws{l}{{k^*}}$ quantifies the mean weight of the label $l$ in the subsignature
$\ssig{k}$. We are interested in subsignatures where $\abs{l}{\ssig{k}}\approx 1$
(label entirely included in $\ssig{k}$) or $\abs{l}{\ssig{k}}\approx 0$ (label
entirely excluded from $\ssig{k}$). This motivates the definition of
$\mems{l}{\ssig{k}}$. Lastly we can calculate the average $\mems{l}{\ssig{k}}$
of $\ssig{k}$, weighted by $N_l$. The more specific a subsignature is to a
subset of labels of $\labelCol{\infoCat}$, the higher $\avmems{\ssig{k}}$ is. 

We say a subsignature is 
\begin{itemize}
    \item\textbf{significant} if it contains at least one label with
        $\pval{\ssig{k}}{l}\le \tpval$
    \item\textbf{mixing} if it is significant and contains at least one label
        with $\aws{\ssig{k}}{l} \ge \taw$
    \item\textbf{selecting} if it is mixing and $\avmems{\ssig{k}} \ge \taw$
\end{itemize}
As in some specific signatures the significant weights are either strictly
positive or negative, we applied to following correction to remove
redundancies:
\begin{itemize}
    \item If either $k^+, k^0$ or $k^-, k^0$ are specific, only consider $k^+$
        or $k^-$.
\end{itemize}

When calculating the above metrics, the choice of $\infoCat$ is
important\todo{rephrase}. E.g. a signature that contains a particular
combination of material types and colors, might not significantly contain either
just the types or just the colors. Vice versa, a choice of
$\infoCat$ that is specific to one label will likely be mutliple coding
for a finer choice of $\infoCat$. The fewer categories are present in the test,
the stronger and more general the signature is. 

