\subsection*{ML Model generation}
\subsubsection*{Data format}
The measurement data exists as a set of files containing absolute intensity
values in the range of \SI{294}{\nano\meter} to \SI{1002}{\nano\meter}. Each
file has associated a meta-file containing information about the measured
sample, and a background file, containing the background measurement for that
particular batch. The relevant entries in the meta file are
\begin{description}
	\item[Type:] Name of measured sample material. Here all PE classes (HDPE,
	LDPE) are grouped as PE.
	\item[Origin:] Either name of manufacturer or location of discovery or
	purchase. Here, all retail plastics are labeled \emph{supermarket}, that
	is we did not distinguish between different retailers.
	\item[Color:] Natural color of the sample
	\item[isPlastic:] Boolean value determining whether sample was plastic or
	%natural
	\item[SampleID:] Unique ID identifying the material sample for each measurement
	\item[BackgroundID:] Unique ID identifying the measurement session 
\end{description}

In the following, $i$ enumerates the set of $N_m$ measurements $m^i \in \mSet$, and $j$
enumerates the set of $N_f$ features (spectral bins) $f_j \in \fSet$.
\subsubsection*{Data Preprocessing}
\label{ssub:preprocForClassification}
A flowchart for the data processing pipeline before classification is shown in
\autoref{sfig:dataprep}.

First, the data (including the background measurement) was interpolated onto a
common spectral axis. Here the number of spectral bins was kept equal to the
mean of number of bins in the overall set. From each measurement the associated background
measurement was subtracted. The data was concatenated into a single \todo{insert
	n x m}matrix. The peak of spectrometer laser is located at
\SI{405}{\nano\meter}. As we don't expect any measurable
excitations below the laser peak, the offset from the zero-line $O^i$ of each
measurement is estimated by taking the median of the data in the spectral range
\SI{294}{\nano\meter} to \SI{400}{\nano\meter}. Similarly, we estimate the noise
level $\eta^i$ of each measurement by calculating the standard deviation in that range.
As we regard any offset of the spectra as systemic, we subtract it from the
data.

To detect the measurements with experimental overexposure, we determined for
each sample the maximum $M^i$ of its smoothed spectrum
where the spectrum was smoothed with a running median of window size
\SI{20}{\nano\meter}. The exposure level was calculated as
$E^i = \frac{O^i}{M^i}$.
Measurements with $E^i < 0.5$ were removed from the data.
\todo{how many were removed?}

The power of the spectrum was calculated as $P^i = \sqrt{\frac{1}{N_f}
	\sum_{j=1\ldots N_f} \qty(\tilde{s}^i_j)^2}$, where $\tilde{s}^i_j$ is the $i$th spectral bin of the running-median smoothed measurement $m^j$. The
signal-to-noise-ratio was calculated as $SNR^i = \frac{P^i}{\eta^i}$.
Measurements with $SNR^i < 2$ were removed from the data. \todo{How many meas.
	were dropped?}

Only measurements for which \emph{SampleID} is known were kept in the data.

For classification, we used the whole spectrum in the range
\SIrange{410}{1002}{\nano\meter} (\emph{long}). For inspection purposes, we
additionally considered the range \SIrange{410}{680}{\nano\meter}
(\emph{short}).

Each measurement was divided by its norm $n^i$ which was calculated via the
Matlab \texttt{trapz} integration function with the spectral range as $x$
dimension.  This step is especially important for SDCM, as otherwise the
regression steps (see \autoref{sssub:methodsSDCM}) might not converge, or
converge very slowly.

In the next step, we performed $25$ \todo{check this number before submitting}
splits of the data into a \emph{preTraining} (\SI{80}{\percent}) and
\emph{validation} (\SI{20}{\percent}) set. The material type and sample-ID
information were used as stratification variables.

Lastly we calculated the median of each spectral bin in the preTraining set and
subtracted it from each measurement in both the preTraining and validation set.
This centers the data at the zero level of each spectral bin. Note that this
might introduce artifacts in data sets that have a different median (e.g. due to
being sampled differently), and thus decrease prediction accuracy in
\emph{external} validation sets.

We additionally performed classifications of spectra and PCA without median
subtraction. As this did not significantly influence the results for PCA, only
the spectra variant is kept in the presented results.
\subsubsection*{Data transformation methods}
\subsubsection*{Classification methods}
We compared the classification results of three different inputs: Using
the processed spectra (\textbf{spectra}), after application of principle component analysis
(\textbf{PCA}) and after application of signal dissection by correlation
maximisation (\textbf{SDCM}). For the interpretation of signatures and principal
components, we used spectra after PCA and SDCM transformation.
\subsection*{Data Interpretation}
